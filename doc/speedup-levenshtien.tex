% This is a sample LaTeX input file.  (Version of 9 April 1986) 
% 
% A '%' character causes TeX to ignore all remaining text on the line, 
% and is used for comments like this one. 

\documentclass[html]{article}    % Specifies the document style. 
%\documentclass[html]{book}    % Specifies the document style. 
\usepackage{graphicx}

\title{A Fast Metric of Document Similarity}  % Declares the document's title. 
\author{Peter Coates}    % Declares the author's name. 
\date{September 13, 2011}   % Deleting this command produces today's date. 

% Define numbers with units to use wherever, e.g., picture size
\def\hpicwidth{1.33in}
\def\hpicheight{4in}

\begin{document}           % End of preamble and beginning of text. 

\maketitle                 % Produces the title. 

{ \em \noindent Abstract: 
This document presents a heuristic for quickly estimating the 
edit-distance of document pairs, by comparing metadata associated with each
file.
Metadata for typical Web pages is from five to twenty bytes in size, 
and edit-distance for a pair of documents can be estimated in less 
than $100\mu$ seconds on generic processors.

Locality-sensitive hashing, shingling, and similar heurisics for
high-speed duplicate detection, tend to excel at making a fast, binary decision
about whether a document duplicates one that has been seen before, but
they have relatively high rates of errors of both the first and second kind. 

This heuristic differs from such algorithms in that it gives a flexible
measurement of the degree of similary of {\em pairs} of texts, with very low
error rates. 
Therefore, it is ideally suited for removing type-one errors from the ouput of
de-duplication algorithms.
It is also capable of detecting more complex relationships between texts
than near-duplication.
For instance, it can be used to recognize that one text embeds another,
to estimate where and/or how diffuse the differences between two documents are, 
or to recognize that two files are textually related, despite being very different.
}

%
% Text here goes above the section-one line
%
% space the draft notice out a little from the rest of the abstract.
\section{Introduction}

Determining whether or not two documents are exactly the same is a linear-time
operation, i.e, two texts can be compared more or less as fast as they 
can be read. 

If you will be testing the same texts repeatedly, and can tolerate a minute
possibility of error, you can do much better.
Texts can be hashed to smaller values, such as 64-bit or 128-bit integers, 
in linear time. 
Thereafter, duplicates can be detected by comparing the fixed-size hashed
metadata. 
If two documents are identical, the hashes will always be identical too 
(although there is a vanishingly small chance of a false positive, i.e., that
non-identical documents will be categorized as identical.)

Not only is this technique computationally efficient, once the hashes have been 
prepared, you no longer need to be in possession of the originals to execute 
the comparison.
Moreover, you can structure your metadata collection for efficient search, e.g.,
by sorting, or by insertion into a hash table or set.
Unfortunately, this strategy only works for documents that are identical down to
the last bit.

Detecting near-duplicates, i.e., pairs of documents that are almost the same, is
a closely related, but more difficult task. 
The need for such near-duplicate detection arises in many applications including:
Web crawling (the Internet is rife with duplicate pages,) merging large 
document databases, detecting plagiarism and other misuse of textual data, etc.

There are numerous techniques for generating metatdata that
achieve for near-duplicates much of what hashing does for exact 
duplicate detection; SimHash, shingling and locality-sensitive hashing are examples.
While these techniques allow incredibly fast detection near duplicates, unlike
hashing, they tend to suffer from high error rates, both for false-positives and
for false negatives (a.k.a, errors of precision and recall, or type-1 and type-2 errors.)
For instance, the duplicate detection algorithm used by Google's Web crawlers,
which must quickly find duplicates among tens of billions of already-seen Web
pages, reportedly has about a 25\% error rate for both false postiives and false
negatives.

\subsection{Levenshtein Distance}

Levenshtein distance ($LD$) is one of many metrics of sequence similarity
that could in principle be used for accurate near-duplicate detection.
Most often applied to text strings, LD is the ``edit distance'' between the
strings.  
I.e., for strings  $S_{1}$ and $S_{2}$,  $LD(S_{1},S_{2})$ is the number of 
single-character additions, deletions, or substitutions, that are required to
convert $S_{1}$ into $S_{2}$. 
Pairs of identical strings have an LD of zero, and the upper bound for LD is 
the length of the longer string\footnote{
	The upper bound is rarely reached unless the files are deliberately constructed
	to be maximally distant, e.g., texts that have no characters in common, or one
	text has length zero, etc.
}.

Edit distance is a great way to define whether two documents are
near-duplicates, but it has two drawbacks: (1) it is not metadata based, meaning
you need possession of the original texts, and (2) it is very slow to compute for
large texts, executing in time and space proportional to the product of the
lengths of the two inputs ({\em O(m*n)}).
For example, a simple implementation executing on a $2.6$ GHZ
laptop processed pairs of $40$-byte strings at approximately $38,000$
pairs/sec, while strings of roughly $100$ times that length, $4$KB, were processed at 
only $3.7$ pairs/sec---about $100^{2}=10,000$, times more slowly. 

\section{Estimating LD} 

Even a rough approximation of LD would be more than enough to determine whether 
two files are near-duplicates. 
Surprisingly, in the most useful cases, it is possible make a
reasonably accurate estimate the LD of pairs of documents very quickly, 
from a small amount of metadata.

The metadata is prepared by compressing the inputs in such a way that the LD of
the compressed strings is approximately proportional to the LD of the originals. 
Because we don't need be able to re-inflate the compressed strings, we
can choose a very lossy algorithm.

The compression procedure described below would typically be set to produce
approximately one bit of metadata\footnote{
Although the algorithm can use any size alphabet for output, we describe here the 
use of only two output characters, 0 and 1.
Two characters are convenient because they are easy to pack into bit-strings,
and provide the most information per bit of output, which is advantageous for 
transmission and storage.
On the other hand, the two-character alphabet is not optimal for processing speed, 
which is primarily a function of the number of characters in the two strings.
There is more information per character in a diverse alphabet.
} for $C$ bytes of text, with $C$
chosen between 25 and 200\footnote{For text---much higher compression rates can be
used for binaries.} The main condition is that the files must be large enough
for the chosen $C$.

A value of C=100, adequate for Web pages, and would yield approximately 100:1
logical compression, but about 800:1 of physical compression. 
Therefore, a 10KB Web page would yield about 13 bytes metadata.

The time savings for compression is large, because LD is a quadratic operation.  
This means that compressed inputs accelerate the computation by the {\em square}
of the compression factor ($C$).
For the example above, with $C=100$, pairs of 10KB Web pages, can be compared 
at a rate of several thousand pairs/second.

\section{The Heuristic}
Two strings $S_{1}$ and $S_{2}$ are processed as follows:

\begin{enumerate}
  	\item {Choose $c$, a compression factor, say, $c=100$.}
  	\item {Choose $n$, a neighborhood size, say, $n=10$.}
	\item {Compress $S_{1}$ and $S_{2}$ into signatures $Sc_{1}$ and $Sc_{2}$,
		using the algorithm given below, parameterized by $c$ and $n$.}
	\item {Execute the standard LD algorithm on results of the previous step, i.e.,
			$LD(Sc_{1},Sc_{2})$.}
	\item {Scale the result by muliplying\footnote{
		When using LD as a metric of gross similarity, it is only the
		ratio of the calculated LD to the upper bound that we actually care 
		about, rather than the integer LD score per se. 
		} by $c$. }
\end{enumerate}

\subsection{Compression}
We want a signature string that has the following properties:
\begin{enumerate}
	\item {
		The signature should be a pseudo-random string of bits that is much smaller
		than the orginal text.
	}
	\item{
		If a substring of the input results in an output bit, no change to the
		input that is more than a fixed number of characters distant should affect
		it. 
	}
	\item{
		A source string should compress to the same sequence of bits,
		regardless of whether it is compressed in isolation, or compressed when 
		embedded in a larger string, except for at most a few bits
		in at either end.
		}
\end{enumerate}

\subsubsection{Compression Steps}
A compressed signature with these properties can be computed as follows:
\begin{enumerate}
  	\item {
  		Choose an integer, $k$, $0\leq k < c$.
	}
	\item{Initialize an accumulator variable, $s$, to the sum of the first $n$
	characters.}
	\item {
		Thereafter, starting at the $n$'th position and proceeding from 
		left to right, one character at a time, until the string is exhausted, 
		at each position, $p$:
		\begin{enumerate}
			\item { Remove the $p-1$'th character from $s$. }
			\item { Append the the $p+n$'th character to $s$.}
			\item{  Compute a pseudo-random integer, $i=s \pmod c$, which is 
				a psuedo-random hash of the current $n$ characters.} 
			\item { If $i=k \pmod c$, emit a bit that is a function of $i$. }
			\item { Otherwise emit nothing.}
		\end{enumerate}
	}	
	\item {The resulting sequence of bits is the compressed string. They would
	ordinarily be stored in a bit field.}
\end{enumerate}

At each position in the input, this procedure will emit either a
pseudo-random bit (app. $1/c$ of the time) or nothing (app. $1-1/c$ of the
time.) Thus, on average, we get a signature that is $1/c$ the size of the input.

If $\Sigma$ is the input alphabet, and $\sigma = |\Sigma|$ is the 
cardinality of the input alphabet, only $\sigma n/c$ distinct sums
will result in output. 

For most purposes, in the useful range of $c$ and $n$, both are small enough
that each of the values in the set, 
$\lbrace i \quad |\quad 0 \leq i < \sigma n, \quad i \equiv k\pmod c \rbrace$, 
can be mapped in a look-up table to 1 or 0\footnote{We are using 0 and 1, but
in general, it can be any alphabet, such as a the set of printable ASCII
characters.}.

Thus, compression is fast requiring only three
arithmetic operations and a lookup, per character of input.
	
Note that given reasonable values of $n$ and $c$, most small differences 
between two inputs do not result in {\em any} difference in the 
corresponding signatures, because the procedure emits a bit on average 
for only $1/c$ of the character positions in the source.

\section{Test Results}
The heuristic was tested primarily on an an earlier version of this \LaTeX \,
document having 13,508 characters and 315 lines. 
Corrupted versions of this file, and other files of the same size, but with
unrelated contents, were also used in some tests.

\subsection{Speed}
With the text pre-loaded into memory on a on 2.6 GHZ processor, plain LD,
applied to a pair of uncompressed 13,508 byte test files, executed at $0.87$
pairs/second.

Similar documents (also preloaded into memory to eliminate file-reading time)
were compressed (on a single thread) at between 7,700 and 10,000 documents per
second (104,000,000 to 135,000,000 characters/second,) depending upon compression rate.

Estimaed LD was computed for the same pairs of files, using metadata prepared
with a range of compression rates, on unrelated 13,508 character text files.
The table blow gives the resulting signature sizes and computation rates.

\vspace {10 mm}
\begin {tabular} {|l|l|l|l|l|r|} \hline \hline
 $C$ 	& $N$ & Pairs/sec & Metadata-bits & Metadata-bytes \\ \hline
 \hline
 
 50 	& 4  & 1,388		& 316 	& 40 \\ \hline
 
 100 	& 4  & 10,000 		& 72 	& 9  \\ \hline
 
 200 	& 4  & 20,000 		& 22  	& 3  \\ \hline
 
\end {tabular} 

\subsection{Accuracy of Estimation}

To test sensitivity and recall, comparisons were made with a range of
compression levels, with the original 13,508 character file, modified in several ways:

\begin{itemize}    
  \item { Random characters were substituted for $1/p$ of the characters in the
  text, for a range of $p$ from  0.01 to 1.5.}
  \item { The text was modified by adding from one to ten blocks of 500
  charaters.}
  \item { 
  	The original 13K text was compared to itself with an equal-length document
  	concatenated onto the beginnning and the end. }
\end{itemize}    



\subsubsection{Numerous Small Changes}
The following table shows the results of estimating LD  on pairs of texts that have a 
range of proportions of single characters changed at random. 
This test simulates many minor editing differences, light formatting, etc. 
Results for a range of compression rates are shown.

The fields are: compression rate, neighborhood size, percentage of random chars that differ, 
the true LD for the documents, the estimated LD, the LD computed for two
completely unrelated files of this size, and the the estimated LD and the LD of 
unrelated files of the same size. 

The LD of the two 13,508 character texts, uncompressed, is 12,430. 
This is a difficult case for estimated LD, because the numerous small changes
affect a large proportion of the neighborhoods in the sample texts.

\vspace {10 mm}
\begin {tabular} {|l|l|l|l|l|l|r|} \hline \hline
 $C$ 	& $N$ & \% Chars Subst'd & True LD	& Est'd LD	& Max LD & Error \\
 \hline \hline
 
 50 	& 4  & 1 				& 136  		& 333 		& 12,430 &	0.0267	\\ \hline
 50 	& 4  & 5 				& 777  		& 1,000 	& 12,430 &	0.0864	\\ \hline
 50 	& 4  & 10 				& 1,438  	& 2,416 	& 12,430 &	0.1943	\\ \hline
 50 	& 4  & 20 				& 2,689 	& 2,350 	& 12,430 &	0.1890	\\ \hline
 50 	& 4  & 30 				& 4,083  	& 3,700 	& 12,430 &	0.2976	\\ \hline
 
 100 	& 4  & 1 				& 119  		& 66 		& 12,430 &	0.0483	\\ \hline
 100 	& 4  & 5  				& 764		& 600 		& 12,430 &	0.2976	\\ \hline
 100 	& 4  & 10 				& 1,451  	& 1,066		& 12,430 &	0.0859	\\ \hline
 100 	& 4  & 20 				& 2,740  	& 1,666 	& 12,430 &	0.1343	\\ \hline
 100 	& 4  & 30 				& 3,981  	& 1,833 	& 12,430 &	0.1478	\\ \hline
 
 200 	& 4  & 1	 			& 115  		& 66 		& 12,430 &	0.0053	\\ \hline
 200 	& 4  & 5	 			& 814  		& 200 		& 12,430 &	0.0161	\\ \hline
 200 	& 4  & 10 				& 1,461  	& 266 		& 12,430 &	0.0322	\\ \hline
 200 	& 4  & 20 				& 2,841  	& 800 		& 12,430 &	0.0645	\\ \hline
 
\end {tabular} 

\subsubsection{A Few Large Changes}
The following table shows the results of estimating LD  on pairs of texts that have a 
several large blocks of random text inserted throughout. 
The original file is 13,508 characters, and the modified files are up to 5000
characters larger. 
This test simulates what might be seen when a plain text
document is converted to one or more HTML page, a Web page is heavily
modified.
Results for a range of compression rates are shown.

The fields are: compression rate, neighborhood size, the number of added blocks, 
the size of the blocks, the actual LD, the estimated LD, 
and the ratio of the error of the estimate to the LD of the original and a  
random text document the size of the modified file.

The smaller the value in the last column, the better the LD estimate.

\vspace {10 mm}
\begin {tabular} {|l|l|l|l|l|l|r|} \hline \hline
 $C$ 	& $N$ & Num-Blocks& Block-Size & True-LD	& Est'd-LD & Est'd-LD/Max \\
 \hline \hline
 
 200 	& 4  & 1 	& 500		& 500  		& 200 		& 	0.0274	\\ \hline
 200 	& 4  & 5 	& 500		& 2,500 	& 2,200 	&	0.0245	\\ \hline
 200 	& 4  & 10 	& 500		& 5,000  	& 6,000 	&	-0.2401	\\ \hline
 
 100 	& 4  & 1 	& 500		& 500  		& 700 		&	-0.0182	\\ \hline
 100 	& 4  & 5  	& 500		& 2,500		& 2,800 	&	-0.0246	\\ \hline
 100 	& 4  & 10 	& 500		& 5,000  	& 4,700   	&	-0.0145	\\ \hline
 
 50 	& 4  & 1	& 500		& 500  		& 700 		&	-0.0182	\\ \hline
 50 	& 4  & 5	& 500 		& 2,500  	& 2,100 	&	0.0327	\\ \hline
 50 	& 4  & 10 	& 500		& 5,000  	& 5,450 	&	-0.0690	\\ \hline
 
\end {tabular} 


\section{Limitations and Considerations}

\subsection{Unrelated Documents}

LD, whether true or estimated, is not a good measure of similarity for files 
that are extremely different, because the LD of random text files of the same 
length is usually far from the theoretical upper bound, which is the length of the 
longer string. 
A significant percentage of the characters in real text will almost always
randomly line up in such a way as to not require an edit operation,or will
align after an insertionor deletion edit is done elswhere. 
Therefore, a small amount of similarity is hard to distinguish from 
background noise.

However, either LD or the estimated LD {\em can } be used to determine the fact
that strings are very different, even though neither effectively measures the amount of 
difference when it is very large.

\subsection{Small Differences }
The majority of small differences between the input files will cause no difference to the output
(because only $1/c$ of the input character positions result in a  non-null character.) 
Therefore, if detecting that there is {\em any} difference is also important, a
separate identity test, such as a cryptographic digest, should be used.

The number of differences, can have a bigger effect on accuracy than the
cumulative size of the differences, because each difference is surrounded by
a neighborhood of width $n-1$ characters that can be affected.

\subsection{Compression Rate}
The value of $c$ affects both speed and sensitivity.  
The frequency with which  the procedure misses small differences, and the
relative effect of the differences that are detected both increase with $c$.

\section{Applications}
The technique may be useful for:
\begin{itemize}
  \item { De-duplicating: detecting variant Web-pages, text files, etc, and
  estimating their textual similarity, as well as verifiying and categorizing
  relationships detected by other duplicate detection algorthms.
  }
  \item {Forensic analysis: speed up searches of disk-drive contents.
		\begin{itemize}
 		\item{ 
  			Multiple versions of known, and therefore uninteresting, executables,
  			system files, or other boiler plate can be identified with a single
  			signature, sidestepping the need for separate cryptographic digests for 
  			every version\footnote{
 			{\em The National Software Reference Library} (NSRL) provides a repository
 			of MD-5 and SHA-1 file signatures of millions of software files for use
 			computer forensic investigations. }. 
  		}
  		\item{
  			Files including arbitrary fragments of text or other data 
  			from a target document can be quickly identified\footnote{
  				As noted elsewhere, if shared fragments of the target 
  				are small relative to the size of the search document,
  				they may get lost in the background noise. 
  				However, a finer resolution can be achieved by breaking large 
  				files into blocks of some maximum size appropriate to the 
  				granularity at which the target text is interesting.}. 
  		}
		\end{itemize}
  }
  \item {Plagiarism detection: Finding occurrences of significant 
  	inclusion of target text embedded in other files.  
  	}
  \item{Intellectual property and data security: A remote service can use these
  		techniques to support double-blind queries concerning whether a significant
  		fragment of the client's content exists in a remote dataset. 
  		
  		The party providing the service reveals nothing about what it has, other
  		than {\em true/false} in response to specific queries about presence.
  		Likewise, the client reveals nothing about the search target, except,
  		obviously, in cases where it is found. 
  		
  		This technique can be used to provide verification to members of the public that 
  		any specific piece of IP is or is not present, even in partial or modified
  		form, in a large data set.
  		
  		The same technique can also be used for data security;
  		Such a service running in the background on each of an institution's workstations can provide a
  		means of detecting the unintentional leakage of sensitive or classified
  		information through cut-and-paste, embedding photos, renaming altered
  		versions of a document, etc, and can do so without either the workstation or the institution 
  		side revealing sensitive information in useful form. 
  	}
\end{itemize}

\section{Conclusion}
After a one-time linear-time pre-processing step, this heuristic estimates the
true Levenshtein edit-distance of text documents many times larger than would be
practical to compare using LD alone.
Estimated LD can be used for detecting near-duplicates, removing false postives,
and otherwise tightening up the output of other duplicate detection algotrithms,
and for detecting more distant relationships between text files, such as
inclusion of one file in another, and highly edited document variants.

The metadata for comparing files is small---half a dozen to a dozen bytes per
10K of text---and computationally cheap to prepare. 
Web pages, articles, and similar text documents of a few tens of kilobytes can
be compared at a rate of many thousands of pairs per second. 

When applied to detecting near-duplicates, errors of both the first and second
kinds are negligible.

%\begin{thebibliography}{Purdom-Brown 85}
% 
 %\bibitem[Mitzenmacher]{mitzen} {\em Network Applications of Bloom Filters}
%	Andrei Broder and Michael Mitzenmacher, Internet Mathematics, Vol. 1, No. 4:
%485-509
%
% \bibitem[Mitzenmacher]{mitzen} {\em Network Applications of Bloom Filters}
%	Andrei Broder and Michael Mitzenmacher, Internet Mathematics, Vol. 1, No. 4:
%485-509
% 
% \end{thebibliography}

\end{document}             % End of document. 
