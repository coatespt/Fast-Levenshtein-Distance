% This is a sample LaTeX input file.  (Version of 9 April 1986) 
% 
% A '%' character causes TeX to ignore all remaining text on the line, 
% and is used for comments like this one. 

\documentclass[html]{article}    % Specifies the document style. 
%\documentclass[html]{book}    % Specifies the document style. 
\usepackage{graphicx}
\usepackage{xfrac}

\title{The Expected Behavior of Estimated Levenshtein}  % Declares the document's title. 
\author{Peter Coates}    % Declares the author's name. 
\date{February 17, 2016}   % Deleting this command produces today's date. 

% Define numbers with units to use wherever, e.g., picture size
\def\hpicwidth{1.33in}
\def\hpicheight{4in}
\def\goedel {Kurt G$\ddot{o}$edel }
\def\neigh {(p,n)}
\def\op {O_{pn}}
\def\opprm {O'_{pn}}

\begin{document}           % End of preamble and beginning of text. 

\maketitle                 % Produces the title. 


%
% Text here goes above the section-one line
%
{\em  \noindent This document is personal notes, not even a draft---don't blame me for the contents.}

% space the draft notice out a little from the rest of the abstract.

\vspace{5mm} 

{\em \noindent 
Abstract: This is where the text of the abstract goes. 
}

\vspace{10mm} 
\section{Introduction}

We are applying extreme compression to input strings so that they can be compared quickly using Levenshtein distance. 
The compression function is applied to each fixed-size neighborhood of the input string yielding
either a character of output or nothing.  
Most neighboroods result in no output, but when an output happens, character chosen is a pseudo-random function of the 
ordered string of characters in the neighborhood.

The compression is extreme because for each neighborhood, the probability of a emitting a character is 
much smaller than the probability of emitting nothing.

Some questions of interest are:
\begin{enumerate}
\item How to characterize the effect on the compressed strings of changes to the input.
\item If you do many trials that consist of modifying k characters of the input and then computing the LD of the original and
the modified version, how closely will the LD's of the compressed signatures cluster around the mean.
\item In real life, it's not usually one character changes that matter---it's word changes and chunks of words.
\item How do you define a confidence interval around near-duplication or relatedness of documents. 
\end{enumerate}

The second of these is tricky. It's a Sorites problem---at what point does a modifed string cease to be a version of the original?  
Even random text will have an LD that is less than the lenght of the strings because some random substrings, particularly
substrings of length one will almost always happen to line up.

Most text of interest is not random at all, although it may be randomly chosen. The relative frequency of letters and common words 
in human languages is very predictable, and some words and groups of words are very common.





\section{Assumptions}

We assume the following:

\begin{itemize}
\item Text input of length $l$. 
\item Input is a text string
\item A neighborhood of the input of size $n$ at point $p$ is $\neigh$.
\item Output alphabet $\alpha$ of size $a$ $\lbrace c_{1}, c_{2}, c_{3}, \dots, c_{a} \rbrace $ 
\item Compressing $\neigh$ yields either an element of $\alpha$ or the null character, $\o$.
\item Probability that compressing $\neigh$ yields a non-null is $1/c$.
\item Hash function of $\neigh$ is ideal---even a one bit difference in $\neigh$ randomizes output completely. 
\item Output for $\neigh$ is $O_{pn}$
\item Output for $\neigh$ after a change is is $O'_{pn}$
\item We make $m$ random one-character changes to the input string.
\end{itemize}
\vspace{5mm} 

\section{Effect of Modifying the Input Sequence}

There are five different situations relating to the application of compression as seen in Table \ref{cases} that 
must sum to $1.0$. 
In two of them, a change to a character in the neighborhood results in a change to the ouptput. 
The sum of these two probabilities is the probability that a one character change will modify the output. 

\begin{table}[h!]
\begin{tabular}{|l|l|l|}
\hline
Case & Expected Output Chars & Change/No-Change\\ \hline
$\op=\o \cap \opprm=\o $ & $(1-1/c)^{2}$ & No \\ \hline 
$\op=\o \cap \opprm \neq \o $ & $\sfrac{(1-1/c)}{c}$ & Yes \\ \hline
$\op \neq \o \cap \opprm=\o $ & $\sfrac{(1-1/c)}{c}$ & Yes \\ \hline
$\op \neq \o \cap \opprm \neq \o \cap \op \neq \opprm $ & $\sfrac{(a-1)}{ac^{2}}$ & Yes \\ \hline
$\op \neq \o \cap \opprm \neq \o \cap \op = \opprm $ & $\sfrac{1}{ac^{2}}$ & No \\ \hline
\end{tabular}
\caption{All Ways That Input Change Can Affect Output}
\label{cases}
\end{table}

As probability that compressing $\neigh$ will result in a null outcome is 
\[
1-1/c
\]

The expected difference in the output for a single neighborhood and a single change is 
\[ \frac{2(1-1/c)}{c} + \frac{a-1}{ac^2} \]


A single change affects $n$ neighborhoods, so the expected effect of a single change is 
\[ E_{n,c} = n \left( \frac{2(1-1/c)}{c} + \frac{a-1}{ac^2} \right) \]


Multiple changes to $\neigh$ are the same as one change in terms of probability of an output, but 
except at the beginning and end of the input string, each change affects n neighborhoods.
Therefore you must discount the effect of collisions of multiple changes in the same neighborhood.

If the input length is $l$ and the number of changes is $m \leq l$ then the proportion of input characters changed is $m/l$
accross the entire input string.
Therefore we discount the expected number of ouput characters affected in poroportion. 
same neighborhood.

\[
	D_{l,m}=1-((m-1)/l)
\]


The adjusted expected number of characters changed per change to an the input characters is the product of these times
multiplied by the total number of input characters changed.
\[
	mE_{n,c}D_{l,m}
\]

\subsection{An Empirical Test}

How much effect should a change to the input have on the output? This is not a simple question because
most inputs of interest natuaral language text, which is not random. 
Likewise, the differences between two texts are rarely random, especially if they are related, which is the 
interesting case. Changes at the word and text-block level are more common than randomly changed 
isolated characters. 
Even so, we start with the simplest case, which is randomly modified characters.

A Java program was prepared to compute the expected LD of a string and a set of modified versions 
of itself, each having a random $2\%$ of the characters swapped for other characters that occur in the document.

The test string was $149,734$ characters of English text from one of the Gutenberg books, and  
of the 500 modified versions had a random $2\%$ of the characters changed for other characters that occur in the document.

The actual LD of the compressed signatures varied from the calculated expected value above by and average
of $-40.362$ characters (i.e, the values tended to be somewhat smaller than expected) with a standard deviation of 
$18.5$ charactes, a min of $-110$ and a max of $10$.

An identical test was executed with a string of the same length that was generated using randomly chosen characters 
from the same set used in the Gutenberg document.
In this case, the average LD deviated from the expected by $31.6$ characters, 
with a standard deviation of $17.8$ characters, a min of =22 and a max of 93.

The test is somewhat unnatural because most text is modified by words and groups of words, not random isolated characters.

\subsection{Noes and What Might Be Wrong}
A couple of things might be wrong with this estimate and need to be checked. 
First, some of the values are very smal

\subsection{Changes At The Word Level}
Any two texts in the same language will show a lower LD than will random character sequences of the same length, if
only because common words like ``the'' are sprinkled throughout both documents.


\section{Confidence Interval Around Judgement of Document Relatedness}

\end{document}             % End of document. 
